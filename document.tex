\documentclass[12pt]{article}

\usepackage{svg}
\svgpath{{img/}}
\usepackage{wrapfig}

\usepackage[T2A]{fontenc}
\usepackage[utf8]{inputenc}
\usepackage[english,russian]{babel}

\usepackage{amsmath}
\usepackage{amsthm, mathrsfs, mathtools, amssymb}
\usepackage{enumitem}

\usepackage{physics}

\usepackage{epigraph}

\usepackage{tikz}
\usepackage{subcaption}
\usetikzlibrary{decorations.pathmorphing}
% для волнистой линии для фотонов создал стиль линии snake arrow
% рисует волну и на конце ее чуть-чуть прямую линию оставляет для стрелки
\tikzset{snake arrow/.style=
	{->,
		decorate,
		decoration={snake,amplitude=.4mm,segment length=2mm,post length=1mm}},
}
\usepackage{caption}
\usepackage{tensor}
\usepackage{float}
\usepackage{multirow}
\usepackage{multicol}
\usepackage{wrapfig}
\usepackage{hyperref}
\hypersetup{
	colorlinks,
	citecolor=black,
	filecolor=black,
	linkcolor=black,
	urlcolor=black
}

\DeclareMathOperator{\supp}{supp}

\usepackage{geometry}
\geometry{verbose,a4paper,tmargin=1cm,bmargin=2cm,lmargin=1.5cm,rmargin=1.5cm}

\newtheoremstyle{example}% name
{0.7cm}% Space above
{0.7cm}% Space below
{\small}% Body font
{}% Indent amount
{\small\scshape}% Theorem head font
{.}% Punctuation after theorem head
{.5em}% Space after theorem head
{}% Theorem head spec (can be left empty, meaning ‘normal’)

\theoremstyle{example}
\newtheorem{example}{Пример}

\theoremstyle{plain}
\newtheorem{theorem}{Теорема}
\newtheorem{corollary}{Следствие}
\newtheorem*{corollary*}{Следствие} 
\newtheorem{lemma}{Лемма}
\newtheorem{utv}{Утверждение}
\newtheorem*{utv*}{Утверждение}

\theoremstyle{definition}
\newtheorem{definition}{Определение}
\newtheorem*{definition*}{Определение}
\newtheorem{question}{Вопрос}

\theoremstyle{remark}
\newtheorem{remark}{Замечание}
\newtheorem*{remark*}{Замечание}
\numberwithin{remark}{section}

\frenchspacing

\usepackage[labelsep=period]{caption}
\captionsetup{font = small}

\DeclareMathAlphabet{\mathbfit}{OML}{cmm}{b}{it}


\newcounter{problem} 
\newenvironment{problem}[1][]
{
	\refstepcounter{problem} 
	\par \vspace{0.7em} \noindent
	\textbf{Задача \theproblem}\ifx&#1&\else\ (#1)\fi. 
}
{
	\vspace{1em}	
}

\newenvironment{solution}
{
	\vspace{0.3em}
	\par\textsc{Решение.}
}
{
	\qed
}

\newcommand{\que}[1]{%
	\subsection{#1}\label{que\thesubsection}
}
\renewcommand{\thesubsection}{\arabic{subsection}}


\begin{document}
	
	\begin{center}
		\Huge \bf	
		\textsc{Оптимальное управление}
		\rule{\textwidth}{0.4pt}
	\end{center}
	\tableofcontents
	
	\que{Постановка задачи Лагранжа в форме Понтрягина. Отсутствие ограничений на фазовые переменные и управления. Допустимые классы функций для фазовых переменных и управлений.}

\subsubsection{Постановка задачи.}

Оптимизируемый функционал имеет вид:
\begin{equation*}
	J(t, \mathbf{x}, \mathbf{u})=\int_{t_0}^{t^1}f_0(t, \mathbf{x}, \mathbf{u})dt\to \min
\end{equation*}

Система состояния объекта $\mathbf{x'}=\mathbf{f}(t, \mathbf{x}, \mathbf{u})$, где $f_0, \mathbf{f}$ - дважды дифференцируемые, $\mathbf{x}$ - дифференцируемые, $\mathbf{u}$ - непрерывные (без дополнительный ограничений на область изменения фазовых переменных и управления).

Краевые условия в любом варианте.

\subsubsection{Решение.}
Лагранжиан с учетом уравнений состояния системы:
\begin{equation*}
	L=f_0+\sum_{i=1}^{n}\lambda_i (x_i'-f_i) = f_0(t, \mathbf{x}, \mathbf{u}) + \mathbf{\lambda}^T(\mathbf{x}'-\mathbf{f})
\end{equation*}
Система уравнений Эйлера:
\begin{equation*}
	\begin{cases}
		L'_{x_i}-\frac{d}{dt}L'_{x'_i}=0,\\
		L'_{u_j}-\frac{d}{dt}L'_{u'_j}=0,\\
	\end{cases}
\end{equation*}
Для заданного выше Лагранжиана имеем:
\begin{align}
	&L'_{x'_i}=\lambda_i(t),&\quad L'_{x_i}=f'_{0\;x_i}-\sum_{k=1}^{n}\lambda_k(t)f'_{k\;x_i},\\
	&L'_{u'_j}=0,&\quad L'_{u_j}=f'_{0\;u_j}-\sum_{k=1}^{n}\lambda_k(t)f'_{k\;u_j}.
\end{align}
Тогда система уравнений Эйлера преобразуется к виду из $n$ дифференциальных и $r$ алгебраических уравнений.

	\que{Постановка задачи ОУ в форме Понтрягина. Допустимые классы функций для фазовых переменных и управлений. Сопряженная система и ее связь с уравнениями Эйлера в канонической форме.}

\subsubsection{Постановка задачи.}

Функционал явно не зависит от времени:
\begin{equation*}
	J(\mathbf{x},\mathbf{u})=\int_{t_0}^{t_1}f_0(\mathbf{x},\mathbf{u}) dt\to\min
\end{equation*}

Система состояния автономна, то есть явно не зависит от времени:
\begin{equation*}
	\mathbf{x}'=\mathbf{f}(\mathbf{x},\mathbf{u}),
\end{equation*}
где $\mathbf{x}$ - кусочно-дифференцируемые функции, $\mathbf{u}$ - кусочно-непрерывные, $f_0, \mathbf{f}$ - непрерывно дифференцируемы по фазовым переменным.

Краевые условия и ограничения на управления:
\begin{equation*}
	\mathbf{x}(t_0)=\mathbf{x}_0,\; \mathbf{x}(t_1) = \mathbf{x}_1,\quad
	\mathbf{u} \in U \in \mathfrak{R}^r
\end{equation*}

Нет ограничений на фазовые переменные, момент времени $t_1$ - не фиксирован.

Решением является допустимый оптимальный процесс $\left(t^*_1,\mathbf{x}^*(t), \mathbf{u}^*(t)\right)$.

\subsubsection{Расширенная система состояния.}
Введем новые переменные:
\begin{equation*}
	x_0(t)=\int_{t_0}^{t} f_0(\mathbf{x},\mathbf{u}) dt,\quad \frac{dx_0(t)}{dt}=f_0(\mathbf{x},\mathbf{u})
\end{equation*}
\begin{equation*}
	\hat{\mathbf{x}} = x_0 + \mathbf{x} = \left[\underbrace{x_0}_{\text{новый член}}, \underbrace{x_1, \dots, x_n}_{\mathbf{x}}\right]^T,\quad
	\hat{\mathbf{f}} = f_0 + \mathbf{f} = \left[\underbrace{f_0}_{\text{новый член}}, \underbrace{f_1, \dots, f_n}_{\mathbf{f}}\right]^T
\end{equation*}
вектор-функция $\hat{\mathbf{f}}$ не зависит от $x_0$.
Тогда расширенная система состояния:
\begin{equation*}
	\hat{\mathbf{x}}'=\hat{\mathbf{f}}'(\mathbf{x},\mathbf{u})
\end{equation*}
Новая форма записи функционала:
\begin{equation*}
	J(\mathbf{x},\mathbf{u}) = x_0(t_1)
\end{equation*}

\subsubsection{Сопряженная система.}
\begin{equation*}
	\frac{d\psi_i}{dt} = -\sum_{j=0}^{n}\frac{\partial f_j}{\partial x_i}(\mathbf{x}, \mathbf{u})\psi_j,
\end{equation*}
получается из уравнений Эйлера в каноническом виде
\begin{equation*}
	\frac{d\psi_i}{dt} = - \frac{\partial H}{\partial x_i},
\end{equation*}
где $H$ - функция Понтрягина имеет вид:
\begin{equation*}
	H(\hat{\mathbf{x}},\hat{\mathbf{\psi}}, \mathbf{u})=\sum_{i=0}^{n}\underbrace{\psi_i(t)}_{\substack{\text{сопряженные}\\\text{переменные}}}\cdot f_i(\mathbf{x},\mathbf{u}).
\end{equation*}

\subsubsection{$\Pi$-система}
Объединенные расширенная система состояний и сопряженная система составляют $\Pi$-систему.
	\que{Условия трансверсальности для задачи Лагранжа в форме Понтрягина при различных условиях на концах. Получить из выражения для первой вариации функционала.}
	\que{Условия трансверсальности для задачи ОУ при различных условиях на концах. Показать получение условий трансверсальности из выражения для первой вариации функционала в задаче Лагранжа и связи между Лагранжианом и функцией Понтрягина.}

См. вопрос \eqref{que3}
	\que{Постановка задачи Больца ОУ. Условия трансверсальности при различных условиях на концах.}
	\que{Формулировка теоремы максимума Понтрягина в простейшей задаче Лагранжа с фиксированными концами и нефиксированным временем на правом конце.}
Пусть 
\begin{itemize}[label=--]
  \item функции $ f_0 $, $ \dot{\mathbf{x}} = \mathbf{f} $ явно не зависят от
времени; 
\item $ \mathbf{x}(t_0) = \mathbf{x}_0 $, $ \mathbf{x}(t_1) = \mathbf{x}_1 $; 
\item момент времени $ t_1 $ не фиксирован;
 \item ограничены управления, но нет ограничений на фазовые переменные.
\end{itemize}

Введём дополнительную фазовую переменную  
\[
  x_0(t) := \int\limits_{t_0}^{t}f_0(\mathbf{x}, \mathbf{u})\,dt, \qquad
  \dot{x}_0(t) = f_0(\mathbf{x}, \mathbf{u}).
\]
Тогда новые вектор-функции $ \hat{\mathbf{x}} $, $ \hat{\mathbf{f}} $ порождают
функцию Гамильтона  
\[
  \mathscr{H}(\hat{\mathbf{x}}, \hat{\mathbf{f}}, \mathbf{u}) = \sum_{i=0}^n
  p_i(t)f^i(\mathbf{x}, \mathbf{u}).
\]
В частности, $ \dot{p}_0(t) = -\frac{\partial \mathscr H}{\partial x_0} = 0 $, то
есть $ p_0 \equiv \mathrm{const} $.

% Пусть также $ M(\hat{\mathbf{x}}, \hat{\mathbf{p}}) := \max\limits_{u\in U} \mathscr H
% $.

\begin{theorem}
  Если $ (\mathbf{x}^\ast(t), \mathbf{u}^{\ast}(t)) $ --- оптимальный процесс
  указанной задачи, то найдётся такая ненулевая вектор-функция $
  \hat{\mathbf{p}}(t) $, удовлетворяющая сопряжённой системе, что 
  \begin{enumerate}
    \item $ \mathscr H(\hat{\mathbf{x}}^{\ast}, \hat{\mathbf{p}}^{\ast},
      \mathbf{u}^\ast) = \max\limits_{u\in U} \mathscr H(\hat{\mathbf{x}}^\ast,
      \hat{\mathbf{p}}^\ast, \mathbf{u}) $.
    \item $ p_0^\ast(t_1) \leqslant 0 $, $ \max\limits_{u\in U} \mathscr H(\hat{\mathbf{x}}^\ast,
      \hat{\mathbf{p}}^\ast, \mathbf{u})\bigr|_{t=t_1} = 0 $.
  \end{enumerate}
\end{theorem}


	\que{Задача быстродействия. Постановка задачи.}
	\que{Задача быстродействия. Формулировка теоремы максимума Понтрягина в этом случае.}
	\que{Линейная задача быстродействия. Условие общности положения для области управлений.}

\begin{definition}
  Задача быстродействия называется \emph{линейной}, если закон движения линеен, а так же область ограничений является замкнутым, ограниченным многогранником, удовлетворяющим также условию общности положения по отношению к системе состояния:
  \[
    \begin{cases}
      \dfrac{d\mathbf{x}}{dt} = A\mathbf{x} + B\mathbf{u}, \\
      \sum_{i=1}^r \alpha_{ji} u_i \leqslant \beta_j, \; j=\overline{1, s},
    \end{cases}
  \]
  где $\mathbf{x} = \left( x_1(t), x_2(t), \dots, x_n(t) \right)^T$,
  $\mathbf{u} = \left( u_1(t), u_2(t), \dots, u_r(t) \right)^T$,
  $A \in \mathbb{R}^{n\times n}, B \in \mathbb{R}^{n\times r}$;
  $\alpha_{ji}, \beta_j \in \mathcal{R}$. 
\end{definition}


\paragraph{Условие общности положения для области управлений}
Пусть $U$ удовлетворяет \emph{условию общности положения по отношению к системе состояний}, т.е. 
$\forall \mathbf{v}\in\mathcal{R}^r$, лежащего на прямой, параллельной какому-либо ребру
многогранника $U$, векторы
$B\mathbf{v}, AB\mathbf{v}, A^2B\mathbf{v}, \dots, A^{n-1}B\mathbf{v}$ -- линейно независимы.

\begin{example}
  Пусть система состояния имеет вид:
  \[
    \begin{cases}
      x_1' = x_2 + u_1,\\
      x_2' = x_3 + u_2,\\
      x_3' = x_1 + u_1,
    \end{cases}
    \Rightarrow
    \mathbf{x}' = \begin{pmatrix} 0 & 1 & 0 \\ 0 & 0 & 1 \\ 1 & 0 & 0 \end{pmatrix} \mathbf{x} +
    \begin{pmatrix} 1 & 0 \\ 0 & 1 \\ 1 & 0 \end{pmatrix} \mathbf{u},
  \]
  и $|u_1| \leqslant 1, |u_2| \leqslant 1$.

  Проверим выполнение условия общности положения для области управлений:
  вектора $\mathbf{v}_1 = \begin{pmatrix} 1 & 0 \end{pmatrix}^T$ и
  $\mathbf{v}_2 = \begin{pmatrix} 0 & 1 \end{pmatrix}^T$ параллельны рёбрам множества $U$.
  Посчитаем $B\mathbf{v}, AB\mathbf{v}, A^2 B \mathbf{v}$:
  \begin{align*}
    \begin{rcases}
      B\mathbf{v}_1 = \begin{pmatrix} 1 & 0 & 1 \end{pmatrix}^T, \\
      AB\mathbf{v}_1 = \begin{pmatrix} 0 & 1 & 1 \end{pmatrix}^T, \\
      A^2 B \mathbf{v}_1 = \begin{pmatrix} 1 & 1 & 0 \end{pmatrix}^T, \\
    \end{rcases} & \Rightarrow \operatorname{det}(B\mathbf{v}, AB\mathbf{v}, A^2 B \mathbf{v}) = -2 \neq 0 \\
    \begin{rcases}
      B\mathbf{v}_2 = \begin{pmatrix} 0 & 1 & 0 \end{pmatrix}^T, \\
      AB\mathbf{v}_2 = \begin{pmatrix} 1 & 0 & 0 \end{pmatrix}^T, \\
      A^2 B \mathbf{v}_2 = \begin{pmatrix} 0 & 0 & 1 \end{pmatrix}^T,
    \end{rcases} &\Rightarrow \operatorname{det}(B\mathbf{v}_2, AB\mathbf{v}_2, A^2 B \mathbf{v}_2) = -1 \neq 0
  \end{align*}

  Следовательно, область управлений удовлетворяет условию общности положения.
\end{example}

	\que{Линейная задача быстродействия. Принцип максимума для линейной задачи быстродействия. Условия, накладываемые на область управлений.}
	\que{Постановка задачи синтеза оптимального управления с позиций принципа максимума Понтрягина. Понятие области управляемости системы.}
	\que{Неавтономные системы. Постановка нестационарной задачи ОУ. Сведение задачи ОУ к случаю автономных систем.}

\begin{definition}
  \emph{Простейшая неавтономная задача ОУ}:
  \[
    \begin{cases}
      J(t, \mathbf{x}, \mathbf{u}) = \int\limits_{t_0}^{t_1} f_0(t, \mathbf{x}, \mathbf{u}) \, dt \to \min, \\
      \mathbf{x}' = \mathbf{f}(t, \mathbf{x}, \mathbf{u}), \\
      \mathbf{x}(t_0) = \mathbf{x}_0, \\
      \mathbf{x}(t_1) = \mathbf{x}_1, \\
      \mathbf{u} \in U \subset \mathcal{R}^r.
    \end{cases}
  \]
\end{definition}

\paragraph{Сведение задачи ОУ к случаю автономных систем.}

Введём еще одну фазовую переменную $x_{n+1}(t) = t$, тогда задача примет вид:
\[
  \begin{cases}
    J(x_{n+1}, \mathbf{x}, \mathbf{u}) = \int\limits_{t_0}^{t_1} f_0(x_{n+1}, \mathbf{x}, \mathbf{u}) \, dt \to \min, \\
    \mathbf{x}' = \mathbf{f}(x_{n+1}, \mathbf{x}, \mathbf{u}), \\
    \mathbf{x}(t_0) = \mathbf{x}_0, \; x_{n+1}(t_0) = t_0, \\
    \mathbf{x}(t_1) = \mathbf{x}_1, \\
    \mathbf{u} \in U \subset \mathcal{R}^r.
  \end{cases}
\]

Условие трансверсальности на правом конце $\psi_{n+1} (t_1) = 0$.

Функция Понтрягина в этом случае:
\[
  H(x_{n+1}, \mathbf{x}, \mathbf{\psi}, \psi_{n+1}, \mathbf{u}) = \sum_{i=0}^n \psi_i(t) \cdot f_i(x_{n+1}, \mathbf{x}, \mathbf{u}) + \psi_{n+1}(t) \cdot 1.
\]

Сопряженная система:
\[
  \begin{cases}
    \dfrac{d \psi_i}{dt} = - \dfrac{\partial H}{\partial x_i}, \, i = \overline{1, n} \\
    \dfrac{d\psi_{n+1}}{dt} = - \dfrac{\partial H}{\partial x_{n+1}} - \dfrac{\partial H}{\partial t} 
  \end{cases}
\]

Таким образом, пришли к постановке заадчи, соответствующей теореме принцип максимума Понтрягина.

% TODO дописать если будет время

\end{document}