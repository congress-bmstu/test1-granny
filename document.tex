\documentclass[12pt]{article}

\usepackage{svg}
\svgpath{{img/}}
\usepackage{wrapfig}

\usepackage[T2A]{fontenc}
\usepackage[utf8]{inputenc}
\usepackage[english,russian]{babel}

\usepackage{amsmath}
\usepackage{amsthm, mathrsfs, mathtools, amssymb}
\usepackage{enumitem}

\usepackage{physics}

\usepackage{epigraph}

\usepackage{tikz}
\usepackage{subcaption}
\usetikzlibrary{decorations.pathmorphing}
% для волнистой линии для фотонов создал стиль линии snake arrow
% рисует волну и на конце ее чуть-чуть прямую линию оставляет для стрелки
\tikzset{snake arrow/.style=
	{->,
		decorate,
		decoration={snake,amplitude=.4mm,segment length=2mm,post length=1mm}},
}
\usepackage{caption}
\usepackage{tensor}
\usepackage{float}
\usepackage{multirow}
\usepackage{multicol}
\usepackage{wrapfig}
\usepackage{hyperref}
\hypersetup{
	colorlinks,
	citecolor=black,
	filecolor=black,
	linkcolor=black,
	urlcolor=black
}

\DeclareMathOperator{\supp}{supp}

\usepackage{geometry}
\geometry{verbose,a4paper,tmargin=1cm,bmargin=2cm,lmargin=1.5cm,rmargin=1.5cm}

\newtheoremstyle{example}% name
{0.7cm}% Space above
{0.7cm}% Space below
{\small}% Body font
{}% Indent amount
{\small\scshape}% Theorem head font
{.}% Punctuation after theorem head
{.5em}% Space after theorem head
{}% Theorem head spec (can be left empty, meaning ‘normal’)

\theoremstyle{example}
\newtheorem{example}{Пример}

\theoremstyle{plain}
\newtheorem{theorem}{Теорема}
\newtheorem{corollary}{Следствие}
\newtheorem*{corollary*}{Следствие} 
\newtheorem{lemma}{Лемма}
\newtheorem{utv}{Утверждение}
\newtheorem*{utv*}{Утверждение}

\theoremstyle{definition}
\newtheorem{definition}{Определение}
\newtheorem*{definition*}{Определение}
\newtheorem{question}{Вопрос}

\theoremstyle{remark}
\newtheorem{remark}{Замечание}
\newtheorem*{remark*}{Замечание}
\numberwithin{remark}{section}

\frenchspacing

\usepackage[labelsep=period]{caption}
\captionsetup{font = small}

\DeclareMathAlphabet{\mathbfit}{OML}{cmm}{b}{it}


\newcounter{problem} 
\newenvironment{problem}[1][]
{
	\refstepcounter{problem} 
	\par \vspace{0.7em} \noindent
	\textbf{Задача \theproblem}\ifx&#1&\else\ (#1)\fi. 
}
{
	\vspace{1em}	
}

\newenvironment{solution}
{
	\vspace{0.3em}
	\par\textsc{Решение.}
}
{
	\qed
}

\newcommand{\que}[1]{%
	\subsection{#1}\label{que\thesubsection}
}
\renewcommand{\thesubsection}{\arabic{subsection}}


\begin{document}
	
	\begin{center}
		\Huge \bf	
		\textsc{Оптимальное управление}
		\rule{\textwidth}{0.4pt}
	\end{center}
	\tableofcontents
	
	\que{Постановка задачи Лагранжа в форме Понтрягина. Отсутствие ограничений на фазовые переменные и управления. Допустимые классы функций для фазовых переменных и управлений.}

\subsubsection{Постановка задачи.}

Оптимизируемый функционал имеет вид:
\begin{equation*}
	J(t, \mathbf{x}, \mathbf{u})=\int_{t_0}^{t^1}f_0(t, \mathbf{x}, \mathbf{u})dt\to \min
\end{equation*}

Система состояния объекта $\mathbf{x'}=\mathbf{f}(t, \mathbf{x}, \mathbf{u})$, где $f_0, \mathbf{f}$ - дважды дифференцируемые, $\mathbf{x}$ - дифференцируемые, $\mathbf{u}$ - непрерывные (без дополнительный ограничений на область изменения фазовых переменных и управления).

Краевые условия в любом варианте.

\subsubsection{Решение.}
Лагранжиан с учетом уравнений состояния системы:
\begin{equation*}
	L=f_0+\sum_{i=1}^{n}\lambda_i (x_i'-f_i) = f_0(t, \mathbf{x}, \mathbf{u}) + \mathbf{\lambda}^T(\mathbf{x}'-\mathbf{f})
\end{equation*}
Система уравнений Эйлера:
\begin{equation*}
	\begin{cases}
		L'_{x_i}-\frac{d}{dt}L'_{x'_i}=0,\\
		L'_{u_j}-\frac{d}{dt}L'_{u'_j}=0,\\
	\end{cases}
\end{equation*}
Для заданного выше Лагранжиана имеем:
\begin{align}
	&L'_{x'_i}=\lambda_i(t),&\quad L'_{x_i}=f'_{0\;x_i}-\sum_{k=1}^{n}\lambda_k(t)f'_{k\;x_i},\\
	&L'_{u'_j}=0,&\quad L'_{u_j}=f'_{0\;u_j}-\sum_{k=1}^{n}\lambda_k(t)f'_{k\;u_j}.
\end{align}
Тогда система уравнений Эйлера преобразуется к виду из $n$ дифференциальных и $r$ алгебраических уравнений.

	\que{Постановка задачи ОУ в форме Понтрягина. Допустимые классы функций для фазовых переменных и управлений. Сопряженная система и ее связь с уравнениями Эйлера в канонической форме.}

\subsubsection{Постановка задачи.}

Функционал явно не зависит от времени:
\begin{equation*}
	J(\mathbf{x},\mathbf{u})=\int_{t_0}^{t_1}f_0(\mathbf{x},\mathbf{u}) dt\to\min
\end{equation*}

Система состояния автономна, то есть явно не зависит от времени:
\begin{equation*}
	\mathbf{x}'=\mathbf{f}(\mathbf{x},\mathbf{u}),
\end{equation*}
где $\mathbf{x}$ - кусочно-дифференцируемые функции, $\mathbf{u}$ - кусочно-непрерывные, $f_0, \mathbf{f}$ - непрерывно дифференцируемы по фазовым переменным.

Краевые условия и ограничения на управления:
\begin{equation*}
	\mathbf{x}(t_0)=\mathbf{x}_0,\; \mathbf{x}(t_1) = \mathbf{x}_1,\quad
	\mathbf{u} \in U \in \mathfrak{R}^r
\end{equation*}

Нет ограничений на фазовые переменные, момент времени $t_1$ - не фиксирован.

Решением является допустимый оптимальный процесс $\left(t^*_1,\mathbf{x}^*(t), \mathbf{u}^*(t)\right)$.

\subsubsection{Расширенная система состояния.}
Введем новые переменные:
\begin{equation*}
	x_0(t)=\int_{t_0}^{t} f_0(\mathbf{x},\mathbf{u}) dt,\quad \frac{dx_0(t)}{dt}=f_0(\mathbf{x},\mathbf{u})
\end{equation*}
\begin{equation*}
	\hat{\mathbf{x}} = x_0 + \mathbf{x} = \left[\underbrace{x_0}_{\text{новый член}}, \underbrace{x_1, \dots, x_n}_{\mathbf{x}}\right]^T,\quad
	\hat{\mathbf{f}} = f_0 + \mathbf{f} = \left[\underbrace{f_0}_{\text{новый член}}, \underbrace{f_1, \dots, f_n}_{\mathbf{f}}\right]^T
\end{equation*}
вектор-функция $\hat{\mathbf{f}}$ не зависит от $x_0$.
Тогда расширенная система состояния:
\begin{equation*}
	\hat{\mathbf{x}}'=\hat{\mathbf{f}}'(\mathbf{x},\mathbf{u})
\end{equation*}
Новая форма записи функционала:
\begin{equation*}
	J(\mathbf{x},\mathbf{u}) = x_0(t_1)
\end{equation*}

\subsubsection{Сопряженная система.}
\begin{equation*}
	\frac{d\psi_i}{dt} = -\sum_{j=0}^{n}\frac{\partial f_j}{\partial x_i}(\mathbf{x}, \mathbf{u})\psi_j,
\end{equation*}
получается из уравнений Эйлера в каноническом виде
\begin{equation*}
	\frac{d\psi_i}{dt} = - \frac{\partial H}{\partial x_i},
\end{equation*}
где $H$ - функция Понтрягина имеет вид:
\begin{equation*}
	H(\hat{\mathbf{x}},\hat{\mathbf{\psi}}, \mathbf{u})=\sum_{i=0}^{n}\underbrace{\psi_i(t)}_{\substack{\text{сопряженные}\\\text{переменные}}}\cdot f_i(\mathbf{x},\mathbf{u}).
\end{equation*}

\subsubsection{$\Pi$-система}
Объединенные расширенная система состояний и сопряженная система составляют $\Pi$-систему.
	\que{Условия трансверсальности для задачи Лагранжа в форме Понтрягина при различных условиях на концах. Получить из выражения для первой вариации функционала.}

\paragraph{Классификация задач вариационного исчисления по виду краевых условий.}
\begin{enumerate}
	\item Закрепленные концы. $\mathbf{y}(x_0)=\mathbf{y}_0,\;\mathbf{y}(x_1)=\mathbf{y}_1$;
	\item Скользящие концы. $x_0, x_1$ фиксированы, а краевые условия $\mathbf{y}(x_0)=\mathbf{y}_0,\;\mathbf{y}(x_1)=\mathbf{y}_1$ \textbf{не} заданы хотя бы на одном конце отрезка $\left[x_0, x_1\right]$;
	\item Свободные концы. Хотя бы одно (или оба) из $x_0, x_1$ не фиксировано. Не заданы краевые условия вида $\mathbf{y}(x_0)=\mathbf{y}_0,\;\mathbf{y}(x_1)=\mathbf{y}_1$ для того значения аргумента на границе участка интегрирования, который не фиксирован. Но при этом ставится достижение концом экстремали некой квазиповерхности $\mathbf{\Gamma}(x_0, \mathbf{y})=\left(\Gamma_1(x_0, \mathbf{y}),\dots,\Gamma_k(x_0, \mathbf{y})\right)^T=\mathbf{0}$ (или аналогично для правого конца $x_1$);
	\item Со свободным временем при заданных значениях $\mathbf{y}(x_0)=\mathbf{y}_0$ и/или $\mathbf{y}(x_1)=\mathbf{y}_1$. Время не фиксировано $x_0$ и/или $x_1$.
\end{enumerate}

Дополнительные соотношения на концах участка интегрирования, учитываемоые при решении задачи Лагранжа, называют \textit{условиями трансверсальности}.

Из курса вариационного исчисления вспомним выражение для первой вариации оптимизируемого функционала:
\begin{multline*}
	\delta J^* = \int_{x_0}^{x_1}\sum_{i=1}^{n} \left(\left(L'_{y_i}-\frac{d}{dx}L'_{y'_i}\right)\delta y_i\right) dx +
	\left.\left[\left(L-\sum_{i=1}^{n}\left(L'_{y'_i}y'_i\right)\right)\delta x + \sum_{i=1}^{n}\left(L'_{y'_i}\delta y_i\right)\right]\right|_{x=x_1} - \\ -
	\left.\left[\left(L-\sum_{i=1}^{n}\left(L'_{y'_i}y'_i\right)\right)\delta x + \sum_{i=1}^{n}\left(L'_{y'_i}\delta y_i\right)\right]\right|_{x=x_0}
\end{multline*}

Из равенства нулю первой вариации вытекает система уравнений Эйлера $L'_{y_i}-\frac{d}{dx}L'_{y'_i}=0$. В данной системе $n$ ОДУ 2-го порядка, для частного решения которой необходимо $2n$ краевых условий.

Далее рассмотрим виды краевых условий:
\begin{enumerate}
	\item Закрепленные концы.
	Имеем $2n$ краевых условий вида $\mathbf{y}(x_0)=\mathbf{y}_0,\;\mathbf{y}(x_1)=\mathbf{y}_1$, при этом все $\delta y_i=0$ на обоих концах участка интегрирования, а также $\left.\delta x\right|_{x=x_0}=\left.\delta x\right|_{x=x_1}=0$, что обнуляет второе и третье слагаемые в первой вариации функционала. Следовательно в случае с закрепленными концами необходимым условие существования экстремума является система уравнений Эйлера с исходными краевыми условиями.
	
	\item Скользящие концы. Во втором и третьем слагаемых равны нулю приращения аргумента на границах, а для равенства нулю первой вариации становится необходимым равенство нулю выражений $\left.L'_{y'_i}\right|_{x=x_0}=0$ или $\left.L'_{y'_i}\right|_{x=x_1}=0$ для тех функций на том конце, где отсутствуют краевые условия. Выражения $\left.L'_{y'_i}\right|_{x=x_0}=0$ или $\left.L'_{y'_i}\right|_{x=x_1}=0$ называют \textit{условиями трансверсальности} или \textit{естественными краевыми условиями}. Общее число краевых условий в этом случае равно $2n$.
	
	\item Свободные концы и доп условия $\mathbf{\Gamma}(x_0, \mathbf{y})=\mathbf{0}$ и/или $\mathbf{\Gamma}(x_1, \mathbf{y})=\mathbf{0}$.
	
	\begin{equation*}
		\left[
		\begin{array}{c}
			\left.\delta x\right|_{x=x_0}\neq0\\
			\left.\delta x\right|_{x=x_1}\neq0
		\end{array}
		\right. \implies
		\left[
		\begin{array}{c}
			\left. \left(L-\sum_{i=1}^{n}\left(L'_{y'_i}y'_i\right)\right)\delta x + \sum_{i=1}^{n}\left(L'_{y'_i}\delta y_i\right) \right|_{x=x_1}=0\\
			\left. \left(L-\sum_{i=1}^{n}\left(L'_{y'_i}y'_i\right)\right)\delta x + \sum_{i=1}^{n}\left(L'_{y'_i}\delta y_i\right) \right|_{x=x_0}=0
		\end{array}
		\right. 
	\end{equation*}
	Недостающие связи между $\delta x$ и $\delta y_i$ получим из $\delta \mathbf{\Gamma}(x_0,\mathbf{y})=\mathbf{0}$ и/или $\delta \mathbf{\Gamma}(x_1,\mathbf{y})=\mathbf{0}$, где
	\begin{equation*}
		\delta \Gamma_j(x,\mathbf{y})=\frac{\partial \Gamma_j}{\partial x}\delta x + \sum_{i=1}^{n}\frac{\partial \Gamma_j}{\partial y_i}\delta y_i.
	\end{equation*}
	\item Со свободным временем при заданных значениях $\mathbf{y}(x_0)=\mathbf{y}_0$ и/или $\mathbf{y}(x_1)=\mathbf{y}_1$.
	Условия трансверсальности примут вид:
	\begin{equation*}
		\left.L - \sum_{i=1}^{n}\left(L'_{y'_i}y'_i\right)\right|_{x=x_0}=0
		\quad\text{и/или}\quad
		\left.L - \sum_{i=1}^{n}\left(L'_{y'_i}y'_i\right)\right|_{x=x_1}=0
	\end{equation*}
\end{enumerate}

\paragraph{В задаче Лагранжа в форме Понтрягина}
Запишем вариацию
\begin{equation*}
		\left.\left[ \left(L-\sum_{i=1}^{n}\left(L'_{x'_i}x'_i\right)\right)\delta t + \sum_{i=1}^{n}\left(L'_{x'_i}\delta x_i\right) \right]\right|_{t=t_1}-
		\left.\left[ \left(L-\sum_{i=1}^{n}\left(L'_{x'_i}x'_i\right)\right)\delta t + \sum_{i=1}^{n}\left(L'_{x'_i}\delta x_i\right) \right]\right|_{t=t_0}=0,
\end{equation*}
где
\begin{align*}
	&L'_{x'_i}\delta x_i=\psi_i(t)\delta x_i\\
	& -H\delta t = \left(L-\sum_{i=1}^{n}\left(L'_{x'_i}x'_i\right)\right)\delta t
\end{align*}
\begin{enumerate}
	\item Закрепленные концы. Условий трансверсальности нет.
	\item Скользящие концы. 
	\begin{equation*}
		\left.\psi_i(t)\right|_{t=t_0} = 0\;\text{или}\;\left.\psi_i(t)\right|_{t=t_1} = 0
	\end{equation*}
	\item Свободные концы.
	\begin{equation*}
		\left.\left(-H\delta t + \sum_{i=1}^{n}\psi_i\delta x_i\right)\right|_{t=t_1}-\left.\left(-H\delta t + \sum_{i=1}^{n}\psi_i\delta x_i\right)\right|_{t=t_0}=0
	\end{equation*}
	\item Со свободным временем при заданных значениях $\mathbf{y}(x_0)=\mathbf{y}_0$ и/или $\mathbf{y}(x_1)=\mathbf{y}_1$.
	\begin{equation*}
		\left.\left(\delta T-H\delta t\right)\right|_{t=t_1}-\left.\left(-\delta\tilde{T}-H\delta t \right)\right|_{t=t_0}=0
	\end{equation*}
	\item В самом общем виде.
	\begin{equation*}
		\left.\left(\delta T-H\delta t+ \sum_{i=1}^{n}\psi_i\delta x_i\right)\right|_{t=t_1}-\left.\left(-\delta\tilde{T}-H\delta t+ \sum_{i=1}^{n}\psi_i\delta x_i \right)\right|_{t=t_0}=0
	\end{equation*}
\end{enumerate}
	\que{Условия трансверсальности для задачи ОУ при различных условиях на концах. Показать получение условий трансверсальности из выражения для первой вариации функционала в задаче Лагранжа и связи между Лагранжианом и функцией Понтрягина.}
	\que{Постановка задачи Больца ОУ. Условия трансверсальности при различных условиях на концах.}
	\que{Формулировка теоремы максимума Понтрягина в простейшей задаче Лагранжа с фиксированными концами и нефиксированным временем на правом конце.}
Пусть 
\begin{itemize}[label=--]
  \item функции $ f_0 $, $ \dot{\mathbf{x}} = \mathbf{f} $ явно не зависят от
времени; 
\item $ \mathbf{x}(t_0) = \mathbf{x}_0 $, $ \mathbf{x}(t_1) = \mathbf{x}_1 $; 
\item момент времени $ t_1 $ не фиксирован;
 \item ограничены управления, но нет ограничений на фазовые переменные.
\end{itemize}

Введём дополнительную фазовую переменную  
\[
  x_0(t) := \int\limits_{t_0}^{t}f_0(\mathbf{x}, \mathbf{u})\,dt, \qquad
  \dot{x}_0(t) = f_0(\mathbf{x}, \mathbf{u}).
\]
Тогда новые вектор-функции $ \hat{\mathbf{x}} $, $ \hat{\mathbf{f}} $ порождают
функцию Гамильтона  
\[
  \mathscr{H}(\hat{\mathbf{x}}, \hat{\mathbf{f}}, \mathbf{u}) = \sum_{i=0}^n
  p_i(t)f^i(\mathbf{x}, \mathbf{u}).
\]
В частности, $ \dot{p}_0(t) = -\frac{\partial \mathscr H}{\partial x_0} = 0 $, то
есть $ p_0 \equiv \mathrm{const} $.

% Пусть также $ M(\hat{\mathbf{x}}, \hat{\mathbf{p}}) := \max\limits_{u\in U} \mathscr H
% $.

\begin{theorem}
  Если $ (\mathbf{x}^\ast(t), \mathbf{u}^{\ast}(t)) $ --- оптимальный процесс
  указанной задачи, то найдётся такая ненулевая вектор-функция $
  \hat{\mathbf{p}}(t) $, удовлетворяющая сопряжённой системе, что 
  \begin{enumerate}
    \item $ \mathscr H(\hat{\mathbf{x}}^{\ast}, \hat{\mathbf{p}}^{\ast},
      \mathbf{u}^\ast) = \max\limits_{u\in U} \mathscr H(\hat{\mathbf{x}}^\ast,
      \hat{\mathbf{p}}^\ast, \mathbf{u}) $.
    \item $ p_0^\ast(t_1) \leqslant 0 $, $ \max\limits_{u\in U} \mathscr H(\hat{\mathbf{x}}^\ast,
      \hat{\mathbf{p}}^\ast, \mathbf{u})\bigr|_{t=t_1} = 0 $.
  \end{enumerate}
\end{theorem}


	\que{Задача быстродействия. Постановка задачи.}
Пусть дана \emph{задача быстродействия}
\begin{itemize}[label=--]
  \item $ f_0 \equiv 1 $;
  \item функция $\dot{\mathbf{x}} = \mathbf{f} $ явно не зависят от
времени; 
\item $ \mathbf{x}(t_0) = \mathbf{x}_0 $, $ \mathbf{x}(t_1) = \mathbf{x}_1 $; 
\item момент времени $ t_1 $ не фиксирован;
 \item ограничены управления, но нет ограничений на фазовые переменные.
\end{itemize}

Введём дополнительную фазовую переменную  
\[
  x_0(t) := \int\limits_{t_0}^{t}f_0(\mathbf{x}, \mathbf{u})\,dt, \qquad
  \dot{x}_0(t) = f_0(\mathbf{x}, \mathbf{u}).
\]
Тогда новые вектор-функции $ \hat{\mathbf{x}} $, $ \hat{\mathbf{f}} $ порождают
функцию Гамильтона  
\[
  \mathscr{H}(\hat{\mathbf{x}}, \hat{\mathbf{f}}, \mathbf{u}) = \sum_{i=0}^n
  p_if^i(\mathbf{x}, \mathbf{u}).
\]
В частности, $ \dot{p}_0(t) = -\frac{\partial \mathscr H}{\partial x_0} = 0 $, то
есть $ p_0 \equiv \mathrm{const} $.

Предположим, что в результате решения $ p_i = 0 $ для всех $ 1 \leqslant 1
\leqslant n $. Отсюда 
вытекает, что и $ p_0 = 0 $. Но по теореме максимума Понтрягина это невозможно. 

	\que{Задача быстродействия. Формулировка теоремы максимума Понтрягина в этом случае.}
% не понял в чём суть вопроса
\begin{theorem}
  Если $ (\mathbf{x}^\ast(t), \mathbf{u}^{\ast}(t)) $ --- оптимальный процесс
  указанной задачи (см. предыдущий вопрос), то найдётся такая ненулевая вектор-функция $
  \hat{\mathbf{p}}(t) $, удовлетворяющая сопряжённой системе, что 
  \begin{enumerate}
    \item $ \mathscr H(\hat{\mathbf{x}}^{\ast}, \hat{\mathbf{p}}^{\ast},
      \mathbf{u}^\ast) = \max\limits_{u\in U} \mathscr H(\hat{\mathbf{x}}^\ast,
      \hat{\mathbf{p}}^\ast, \mathbf{u}) $.
    \item $ p_0^\ast(t_1) \leqslant 0 $, $ \max\limits_{u\in U} \mathscr H(\hat{\mathbf{x}}^\ast,
      \hat{\mathbf{p}}^\ast, \mathbf{u})\bigr|_{t=t_1} = 0 $.
  \end{enumerate}
\end{theorem}


	\que{Линейная задача быстродействия. Условие общности положения для области управлений.}
	\que{Линейная задача быстродействия. Принцип максимума для линейной задачи быстродействия. Условия, накладываемые на область управлений.}

\begin{definition}
  Задача быстродействия называется \emph{линейной}, если закон движения линеен, а так же область ограничений является замкнутым, ограниченным многогранником, удовлетворяющим также условию общности положения по отношению к системе состояния:
  \[
    \begin{cases}
      \dfrac{d\mathbf{x}}{dt} = A\mathbf{x} + B\mathbf{u}, \\
      \sum_{i=1}^r \alpha_{ji} u_i \leqslant \beta_j, \; j=\overline{1, s},
    \end{cases}
  \]
  где $\mathbf{x} = \left( x_1(t), x_2(t), \dots, x_n(t) \right)^T$,
  $\mathbf{u} = \left( u_1(t), u_2(t), \dots, u_r(t) \right)^T$,
  $A \in \mathbb{R}^{n\times n}, B \in \mathbb{R}^{n\times r}$;
  $\alpha_{ji}, \beta_j \in \mathcal{R}$. 
\end{definition}

\paragraph{Функция Понтрягина}
\[
  H(\mathbf{x}, \mathbf{\psi}, \mathbf{u}) = \mathbf{\psi}^T A \mathbf{x} + \mathbf{\psi}^T B \mathbf{u} - 1,
\]
Эта функция линейна по управлению, следовательно, максимум по управлению будет достигаться на границах области $U$.


\paragraph{Сопряженная система}
\[
  \mathbf{\psi} = - \dfrac{\partial H}{\partial \mathbf{x}} - \dfrac{\partial (- \mathbf{\psi}^T A \mathbf{x})}{\partial \mathbf{x}} = - A^T \mathbf{\psi}.
\]
Заметим, что решение этой нормальной системы ДУ существует и не зависит от управления.

\begin{theorem}
  Для любого нетривиального решения $\mathbf{\psi}(t)$ сопряженной системы
  $\mathbf{\psi}' = -A^T \mathbf{\psi}$ в линейной задаче быстродействия соотношение
  $\max_{\mathbf{u} \in U} \left\{ F(\mathbf{\psi}, \mathbf{u} \right\} =
  F(\mathbf{\psi}^*, \mathbf{u}^*)$ однозначно определит оптимальное управление $\mathbf{u}^*(t)$,
  причём оно будет кусочно-постоянным, а значения оптимального управления в точках его
  непрерывности являются вершинами многогранника $U$ (многогранник $U$ при этом удовлетворяет
  уcловию общности положения по отношению к системе состояния объекта).
\end{theorem}

\begin{theorem}
  Если существует хотя бы одно управление, переводящее систему из точки
  $\mathbf{x}(t_0) = \mathbf{x}_0$ в точку $\mathbf{x}(t_1) = \mathbf{x}_1$, то существует и
  отпиматльное по быстродействию управление.

  В случае линейной системы состояния оптимальное управление в задаче
  быстродействия единственно и оно соответствует принципу максимума Понтрягина. В
  этом случае принцип максимума не только необходимое, но и достаточное условие
  существования оптимального управления.
\end{theorem}

	\que{Постановка задачи синтеза оптимального управления с позиций принципа максимума Понтрягина. Понятие области управляемости системы.}

\begin{definition}
  Задача синтеза управления -- отыскание управления $\mathbf{u}^* (\mathbf{x})$, где
  $\mathbf{u}^*(\mathbf{x})$ называется \emph{синтезирующей функцией} или \emph{обратной связью}.
\end{definition}

(Пытаемся найти вид управления до того, как узнали краевые условия)

\begin{definition}
  Область управляемости -- область таких значений краевых условий, при которых существует
  управление, удовлетворяющее системе состояния, а решение $\mathbf{x}$ -- удовлетворяющее этим
  краевым.
\end{definition}




	\que{Неавтономные системы. Постановка нестационарной задачи ОУ. Сведение задачи ОУ к случаю автономных систем.}
\end{document}