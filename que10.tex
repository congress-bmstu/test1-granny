\que{Линейная задача быстродействия. Принцип максимума для линейной задачи быстродействия. Условия, накладываемые на область управлений.}

\begin{definition}
  Задача быстродействия называется \emph{линейной}, если закон движения линеен, а так же область ограничений является замкнутым, ограниченным многогранником, удовлетворяющим также условию общности положения по отношению к системе состояния:
  \[
    \begin{cases}
      \dfrac{d\mathbf{x}}{dt} = A\mathbf{x} + B\mathbf{u}, \\
      \sum_{i=1}^r \alpha_{ji} u_i \leqslant \beta_j, \; j=\overline{1, s},
    \end{cases}
  \]
  где $\mathbf{x} = \left( x_1(t), x_2(t), \dots, x_n(t) \right)^T$,
  $\mathbf{u} = \left( u_1(t), u_2(t), \dots, u_r(t) \right)^T$,
  $A \in \mathbb{R}^{n\times n}, B \in \mathbb{R}^{n\times r}$;
  $\alpha_{ji}, \beta_j \in \mathcal{R}$. 
\end{definition}

\paragraph{Функция Понтрягина}
\[
  H(\mathbf{x}, \mathbf{\psi}, \mathbf{u}) = \mathbf{\psi}^T A \mathbf{x} + \mathbf{\psi}^T B \mathbf{u} - 1,
\]
Эта функция линейна по управлению, следовательно, максимум по управлению будет достигаться на границах области $U$.


\paragraph{Сопряженная система}
\[
  \mathbf{\psi} = - \dfrac{\partial H}{\partial \mathbf{x}} - \dfrac{\partial (- \mathbf{\psi}^T A \mathbf{x})}{\partial \mathbf{x}} = - A^T \mathbf{\psi}.
\]
Заметим, что решение этой нормальной системы ДУ существует и не зависит от управления.

\begin{theorem}
  Для любого нетривиального решения $\mathbf{\psi}(t)$ сопряженной системы
  $\mathbf{\psi}' = -A^T \mathbf{\psi}$ в линейной задаче быстродействия соотношение
  $\max_{\mathbf{u} \in U} \left\{ F(\mathbf{\psi}, \mathbf{u} \right\} =
  F(\mathbf{\psi}^*, \mathbf{u}^*)$ однозначно определит оптимальное управление $\mathbf{u}^*(t)$,
  причём оно будет кусочно-постоянным, а значения оптимального управления в точках его
  непрерывности являются вершинами многогранника $U$ (многогранник $U$ при этом удовлетворяет
  уcловию общности положения по отношению к системе состояния объекта).
\end{theorem}

\begin{theorem}
  Если существует хотя бы одно управление, переводящее систему из точки
  $\mathbf{x}(t_0) = \mathbf{x}_0$ в точку $\mathbf{x}(t_1) = \mathbf{x}_1$, то существует и
  отпиматльное по быстродействию управление.

  В случае линейной системы состояния оптимальное управление в задаче
  быстродействия единственно и оно соответствует принципу максимума Понтрягина. В
  этом случае принцип максимума не только необходимое, но и достаточное условие
  существования оптимального управления.
\end{theorem}
