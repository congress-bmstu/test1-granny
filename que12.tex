\que{Неавтономные системы. Постановка нестационарной задачи ОУ. Сведение задачи ОУ к случаю автономных систем.}

\begin{definition}
  \emph{Простейшая неавтономная задача ОУ}:
  \[
    \begin{cases}
      J(t, \mathbf{x}, \mathbf{u}) = \int\limits_{t_0}^{t_1} f_0(t, \mathbf{x}, \mathbf{u}) \, dt \to \min, \\
      \mathbf{x}' = \mathbf{f}(t, \mathbf{x}, \mathbf{u}), \\
      \mathbf{x}(t_0) = \mathbf{x}_0, \\
      \mathbf{x}(t_1) = \mathbf{x}_1, \\
      \mathbf{u} \in U \subset \mathcal{R}^r.
    \end{cases}
  \]
\end{definition}

\paragraph{Сведение задачи ОУ к случаю автономных систем.}

Введём еще одну фазовую переменную $x_{n+1}(t) = t$, тогда задача примет вид:
\[
  \begin{cases}
    J(x_{n+1}, \mathbf{x}, \mathbf{u}) = \int\limits_{t_0}^{t_1} f_0(x_{n+1}, \mathbf{x}, \mathbf{u}) \, dt \to \min, \\
    \mathbf{x}' = \mathbf{f}(x_{n+1}, \mathbf{x}, \mathbf{u}), \\
    \mathbf{x}(t_0) = \mathbf{x}_0, \; x_{n+1}(t_0) = t_0, \\
    \mathbf{x}(t_1) = \mathbf{x}_1, \\
    \mathbf{u} \in U \subset \mathcal{R}^r.
  \end{cases}
\]

Условие трансверсальности на правом конце $\psi_{n+1} (t_1) = 0$.

Функция Понтрягина в этом случае:
\[
  H(x_{n+1}, \mathbf{x}, \mathbf{\psi}, \psi_{n+1}, \mathbf{u}) = \sum_{i=0}^n \psi_i(t) \cdot f_i(x_{n+1}, \mathbf{x}, \mathbf{u}) + \psi_{n+1}(t) \cdot 1.
\]

Сопряженная система:
\[
  \begin{cases}
    \dfrac{d \psi_i}{dt} = - \dfrac{\partial H}{\partial x_i}, \, i = \overline{1, n} \\
    \dfrac{d\psi_{n+1}}{dt} = - \dfrac{\partial H}{\partial x_{n+1}} - \dfrac{\partial H}{\partial t} 
  \end{cases}
\]

Таким образом, пришли к постановке заадчи, соответствующей теореме принцип максимума Понтрягина.

% TODO дописать если будет время
