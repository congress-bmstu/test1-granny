\que{Задача быстродействия. Формулировка теоремы максимума Понтрягина в этом случае.}
% не понял в чём суть вопроса
\begin{theorem}
  Если $ (\mathbf{x}^\ast(t), \mathbf{u}^{\ast}(t)) $ --- оптимальный процесс
  указанной задачи (см. предыдущий вопрос), то найдётся такая ненулевая вектор-функция $
  \hat{\mathbf{p}}(t) $, удовлетворяющая сопряжённой системе, что 
  \begin{enumerate}
    \item $ \mathscr H(\hat{\mathbf{x}}^{\ast}, \hat{\mathbf{p}}^{\ast},
      \mathbf{u}^\ast) = \max\limits_{u\in U} \mathscr H(\hat{\mathbf{x}}^\ast,
      \hat{\mathbf{p}}^\ast, \mathbf{u}) $.
    \item $ p_0^\ast(t_1) \leqslant 0 $, $ \max\limits_{u\in U} \mathscr H(\hat{\mathbf{x}}^\ast,
      \hat{\mathbf{p}}^\ast, \mathbf{u})\bigr|_{t=t_1} = 0 $.
  \end{enumerate}
\end{theorem}

