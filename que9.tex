\que{Линейная задача быстродействия. Условие общности положения для области управлений.}

\begin{definition}
  Задача быстродействия называется \emph{линейной}, если закон движения линеен, а так же область ограничений является замкнутым, ограниченным многогранником, удовлетворяющим также условию общности положения по отношению к системе состояния:
  \[
    \begin{cases}
      \dfrac{d\mathbf{x}}{dt} = A\mathbf{x} + B\mathbf{u}, \\
      \sum_{i=1}^r \alpha_{ji} u_i \leqslant \beta_j, \; j=\overline{1, s},
    \end{cases}
  \]
  где $\mathbf{x} = \left( x_1(t), x_2(t), \dots, x_n(t) \right)^T$,
  $\mathbf{u} = \left( u_1(t), u_2(t), \dots, u_r(t) \right)^T$,
  $A \in \mathbb{R}^{n\times n}, B \in \mathbb{R}^{n\times r}$;
  $\alpha_{ji}, \beta_j \in \mathcal{R}$. 
\end{definition}


\paragraph{Условие общности положения для области управлений}
Пусть $U$ удовлетворяет \emph{условию общности положения по отношению к системе состояний}, т.е. 
$\forall \mathbf{v}\in\mathcal{R}^r$, лежащего на прямой, параллельной какому-либо ребру
многогранника $U$, векторы
$B\mathbf{v}, AB\mathbf{v}, A^2B\mathbf{v}, \dots, A^{n-1}B\mathbf{v}$ -- линейно независимы.

\begin{example}
  Пусть система состояния имеет вид:
  \[
    \begin{cases}
      x_1' = x_2 + u_1,\\
      x_2' = x_3 + u_2,\\
      x_3' = x_1 + u_1,
    \end{cases}
    \Rightarrow
    \mathbf{x}' = \begin{pmatrix} 0 & 1 & 0 \\ 0 & 0 & 1 \\ 1 & 0 & 0 \end{pmatrix} \mathbf{x} +
    \begin{pmatrix} 1 & 0 \\ 0 & 1 \\ 1 & 0 \end{pmatrix} \mathbf{u},
  \]
  и $|u_1| \leqslant 1, |u_2| \leqslant 1$.

  Проверим выполнение условия общности положения для области управлений:
  вектора $\mathbf{v}_1 = \begin{pmatrix} 1 & 0 \end{pmatrix}^T$ и
  $\mathbf{v}_2 = \begin{pmatrix} 0 & 1 \end{pmatrix}^T$ параллельны рёбрам множества $U$.
  Посчитаем $B\mathbf{v}, AB\mathbf{v}, A^2 B \mathbf{v}$:
  \begin{align*}
    \begin{rcases}
      B\mathbf{v}_1 = \begin{pmatrix} 1 & 0 & 1 \end{pmatrix}^T, \\
      AB\mathbf{v}_1 = \begin{pmatrix} 0 & 1 & 1 \end{pmatrix}^T, \\
      A^2 B \mathbf{v}_1 = \begin{pmatrix} 1 & 1 & 0 \end{pmatrix}^T, \\
    \end{rcases} & \Rightarrow \operatorname{det}(B\mathbf{v}, AB\mathbf{v}, A^2 B \mathbf{v}) = -2 \neq 0 \\
    \begin{rcases}
      B\mathbf{v}_2 = \begin{pmatrix} 0 & 1 & 0 \end{pmatrix}^T, \\
      AB\mathbf{v}_2 = \begin{pmatrix} 1 & 0 & 0 \end{pmatrix}^T, \\
      A^2 B \mathbf{v}_2 = \begin{pmatrix} 0 & 0 & 1 \end{pmatrix}^T,
    \end{rcases} &\Rightarrow \operatorname{det}(B\mathbf{v}_2, AB\mathbf{v}_2, A^2 B \mathbf{v}_2) = -1 \neq 0
  \end{align*}

  Следовательно, область управлений удовлетворяет условию общности положения.
\end{example}
